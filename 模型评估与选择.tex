\section{模型评估与选择}
\subsection{经验误差与拟合}
\myitem 错误率:m个样本中有a个样本分类错误,则错误率$E=a/m$.
\myitem 精度:$1-a/m$
\myitem 误差:学习器的实际预测输出与样本的真实输出之间的差异;训练集上的叫\underline{训练误差}/\underline{经验误差},新样本上的叫\underline{泛化误差};
\myitem 过拟合(更容易遇到的情况):将个例的特殊性错误地视为样本的普遍性;欠拟合:没有学好.
\par 过拟合无法避免(N=NP问题尚未证明),只能缓解或减小其风险
\myitem 模型选择:在多种学习算法中,选择合适的算法中的合适的参数配置.
\subsection{评估方法}
\myitem 测试集测试学习器对新样本的判断力,将测试误差近似于泛化误差.\underline{测试集尽可能的不出现在训练集中}.(老师更希望学生有举一反三的能力)
\myitem \textbf{如果只有一个数据集,既要训练又要测试,如何解决?}
\subsubsection{留出法}
直接将数据集D划分为两个互斥的集合,其中一个集合作为训练集S,另一个作为测试集T.在S上训练出模型后用T来评估其测试误差.
\myitem 训练/测试集的划分要尽可能保持数据分布的一致性(分类任务时,至少要保持样本的类别比例相似)采样角度上看,则保留类别比例的采样方式通常称为\underline{分层采样}\par
例如:通过对D进行分层采样而获得含70\%样本的训练集S和含30\%样本的测试集T,
若D包含500个正例、500个反例,则分层采样得到的S应包含350个正例、350个反例,而T则包含150个正例和150个反例;
若S、T中样本类别比例差别很大,则误差估计将由于训练/测试数据分布的差异而产生偏差.
\myitem 单次使用留出法得到的估计结果往往不够稳定可靠,在使用留出法时,一般要采用若干次随机划分、重复进行实验评估后取平均值作为留出法的评估结果.\par
例如进行100次随机划分,每次产生一个训练/测试集用于实验评估,100次后就得到100个结果,而留出法返回的则是这100个结果的平均.
\myitem 局限性:若令训练集S包含绝大多数样本,则训练出的模型可能更接近于用D训练出的模型,但由于T比较小,评估结果可能不够稳定准确;
若令测试集T多包含一些样本,则训练集S与D差别更大了,被评估的模型与用D训练出的模型相比可能有较大差别,从而降低了评估结果的保真性(fidelity).这个问题没有完美的解决方案,常见做法是将大约2/3~4/5的样本用于训练,剩余样本用于测试.
\subsubsection{交叉验证法}
将数据集分为k个相似的子集,每一折交叉验证时,k-1为训练集,剩余的一个为测试集,共k次不同的测试集,为k折交叉验证法.
交叉验证法同样需要进行随机划分,例如10次10折交叉验证法和100次留出法都需要进行100次训练/测试.
\par \textbf{\heiti 留一法(LOO)},数据集共m个,当k=m时,为留一法.训练出来的模型会更准确,但计算开销较大.
\subsubsection{自助法(Bootstrapping)}
有放回的重复随机抽样,执行m次得到m个样本的数据集$D'$.其中,某些样本始终不会被抽中的概率为
$\lim_{m->\infty}(1-\frac{1}{m})^m\mapsto\frac{1}{e}\approx 0.368$
自助法适用于小样本.
\subsubsection{小结}
\begin{table}[ht]
\centering
\begin{tabular}{lcccc}
\toprule
\textbf{方法} & \textbf{数据划分方式} & \textbf{数据利用率} & \textbf{计算成本} & \textbf{适用场景} \\
\midrule
留出法 & 单次划分训练集/测试集 & 低 & 低 & 大数据快速验证 \\
交叉验证 & 多次划分(\(k\)个子集轮流验证) & 高 & 中 & 中等数据模型调优 \\
留一法 & 每次留1个样本验证 & 最高 & 高 & 小样本高精度评估 \\
自助法 & 有放回重采样生成多样本集 & 灵活 & 中 & 小样本统计推断、不确定性估计 \\
\bottomrule
\end{tabular}
\end{table}

\textbf{\heiti 选择建议}
\begin{itemize}[noitemsep]
    \item 数据量大且需快速验证:留出法。
    \item 中等数据调参:\(k\)-折交叉验证(如5折或10折)。
    \item 小样本高精度需求:留一法(但需权衡计算成本)。
    \item 统计量分布估计或小样本:自助法。
\end{itemize}
\subsubsection{调参与最终模型}
\newpage
\subsection{性能度量}

\subsubsection{错误率与精度}
