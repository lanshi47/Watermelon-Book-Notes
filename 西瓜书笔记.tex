\documentclass[a4paper,12pt,UTF8,fontset=none]{ctexbook} % 使用ctexbook文档类
\usepackage{geometry} % 页面设置
\usepackage{xeCJK}
\usepackage{enumitem}
\usepackage{xcolor} % Color support for listings
\usepackage{graphicx}
\usepackage{float}
\usepackage{pdfpages}
\usepackage{amsmath, amsthm, amssymb}
\usepackage{listings}
\usepackage{longtable}
\usepackage{booktabs} % 添加booktabs宏包以支持三线表
\usepackage{makecell}
\usepackage{tabularx} % 新增tabularx宏包用于自动调整列宽
\usepackage{array}      % 列格式控制
\usepackage{caption}    % 表格标题格式
\usepackage{placeins} 
\usepackage{newtxtext,newtxmath}
\usepackage{fontspec}
\usepackage{titlesec}
\usepackage{bm} % 加粗符号
\usepackage{background} % 添加水印宏包
\usepackage{hyperref} % 超链接引用
\usepackage{cleveref}
\usepackage{fancyhdr}  % 导入页眉页脚宏包


\newtheorem{theorem}{定理}[section] % 定理编号按章节
\newtheorem{proofstep}{步骤}[theorem] % 证明步骤编号依附定理

% 自定义关键项高亮命令
\newcommand{\highlight}[1]{\textcolor{red}{\boxed{#1}}}
\backgroundsetup{ % 水印参数设置
    scale=3,          % 水印文字大小(可调)
    angle=0,         % 水印旋转角度
    opacity=0.25,     % 透明度(0-1)
    contents={\includegraphics[width=0.5\textwidth]{static/pdfs/水印.pdf}} % 改为正确的水印文件
    % contents={lanshi}
}
% 设置英文主字体为 Times New Roman
\setmainfont{Times New Roman}[Path=D:/Program Files/MiKTeX/fonts/custom/, Extension=.otf]

% 设置英文粗体字体为 Times New Roman Bold
\newfontfamily\enboldfont{Times New Roman Bold}[Path=D:/Program Files/MiKTeX/fonts/custom/, Extension=.otf]

% 设置中文正文字体为宋体
\setCJKmainfont{SimSun}[Path=D:/Program Files/MiKTeX/fonts/custom/, Extension=.ttf]

% 设置中文黑体字体
\setCJKfamilyfont{zhhei}{SimHei}[Path=D:/Program Files/MiKTeX/fonts/custom/, Extension=.ttf]
\newcommand{\heiti}{\CJKfamily{zhhei}}


% 定义计数器
\newcounter{mycounter}

% 定义命令用于显示编号
\newcommand{\myitem}{%
  \par \stepcounter{mycounter}%
  \textbf{\themycounter.}~%
}
% 设置页眉页脚样式
\pagestyle{fancy}     
\fancyhead[LE]{\leftmark}    % 偶数页左:章标题
\fancyhead[RO]{\rightmark}   % 奇数页右:节标题
\fancyhead[LO]{\thepage}     % 奇数页左:页码
\fancyhead[RE]{\thepage}     % 偶数页右:页码
\renewcommand{\headrulewidth}{0.4pt} % 页眉下划线厚度
% 设置代码块样式
\lstset{
    basicstyle=\ttfamily,
    columns=fullflexible,
    frame=single,
    breaklines=true,
    postbreak=\mbox{\textcolor{red}{$\hookrightarrow$}\space},
    language=SQL,
    keywordstyle=\color{blue},
    commentstyle=\color{gray},    
    rulecolor=\color{black!30},%边框颜色
    stringstyle=\color{red},
    escapeinside={\%*}{*)},
    showstringspaces=false,
    captionpos=b % 设置标题位置, b表示在底部
}
\lstset{
    basicstyle=\ttfamily,
    keywordstyle=\color{blue},
    commentstyle=\color{green},
    stringstyle=\color{red},
    numbers=left,
    numberstyle=\tiny,
    frame=single,
    backgroundcolor=\color{white},
    language=Matlab
}
\geometry{left=2.5cm,right=2.5cm,top=2.2cm,bottom=2.2cm} % 页边距

\begin{document}

% 制作封面页
\begin{titlepage}
    \centering
    \vspace*{\fill}
    {\LARGE\bfseries 人工智能学习笔记\par}
    \vspace{2cm}
    {\Huge\bfseries 西瓜书学习笔记\par}
    \vspace{2cm}
    {\Large 烂石\par}
    \vspace{1cm}
    {\large \today \par}
    \vspace{4cm}
    \includegraphics[width=0.5\textwidth]{static/images/logo.png}
    \vspace*{\fill}
    \thispagestyle{empty} % 封面不显示页码
    \newpage
\end{titlepage}
\section{绪论}
\subsection{引言}
略
\subsection{基本术语}
\begin{enumerate}[label=(\roman*)]
    \item 样本/示例-sample/instance
    \item 训练集-training data
    \item 测试集-testing data
    \item 标记-label 
    \item 样例-example:拥有label的instance
    \item 泛化-generalization
\end{enumerate}
\subsection{假设空间}
\subsubsection{科学推理}
\begin{itemize}
    \item 归纳--induction
    从具体的事实归结出一般性规律
    \item 演绎--deduction
    从基础原理推演出具体状况
\end{itemize}
\subsubsection{归纳学习--inductive learning}
\begin{itemize}
    \item 广义归纳学习
    \item 狭义归纳学习--概念学习
    eg:布尔概念学习
\end{itemize}
\subsubsection{版本空间--version space}
即存在着一个与训练集一致的"假设集合"
\subsection{归纳偏好}
有多个与训练集一致的假设,但测试新样本时有不同的输出结果,那么采用哪种模型(假设)?
\subsubsection{"奥卡姆剃刀"原则}
若有多个假设与观察一致,则选最简单的那个.
利用什么原则,取决于算法能否获得更好的性能,泛化能力是否更强
\subsubsection{NFL(No Free Lunch Theorem)定理--"没有免费的午餐"定理}

\begin{theorem}[No Free Lunch 定理]\label{thm:nfl}
    对于所有学习算法 $\mathcal{L}_a$ 和 $\mathcal{L}_b$,在均匀分布的目标函数空间下,它们的训练外误差满足:
    \[
    \sum_f E_{ote}(\mathcal{L}_a|X,f) = \sum_f E_{ote}(\mathcal{L}_b|X,f)
    \]
\end{theorem}
\begin{proof}
    \begin{proofstep}[定义与假设]\label{step:def}
    假设样本空间 $\mathcal{X}$ 和假设空间 $\mathcal{H}$ 是离散的。定义:
    \[
    E_{ote}(\mathcal{L}_a|X,f) = \sum_{h \in \mathcal{H}} \sum_{x \in \mathcal{X} - X} P(x) \cdot \mathbb{I}(h(x) \neq f(x)) \cdot P(h|X,\mathcal{L}_a)
    \]
    其中 $\mathbb{I}(\cdot)$ 为指示函数。
    \end{proofstep}
    
    \begin{proofstep}[总误差求和]\label{step:sum}
    对所有目标函数求和:
    \begin{equation}
    \sum_f E_{ote}(\mathcal{L}_a|X,f) = \sum_f \sum_h \sum_{x \in \mathcal{X}- X} P(x)\mathbb{I}(h(x)\neq f(x))P(h|X,\mathcal{L}_a)
    \end{equation}
    \end{proofstep}
    
    \begin{proofstep}[交换求和顺序]\label{step:swap}
    将 $\sum_f$ 移至内部:
    \begin{align}
    &= \sum_{x \in \mathcal{X}- X} P(x) \sum_h P(h|X,\mathcal{L}_a) \nonumber \\
    &\quad \times \underbrace{\sum_f \mathbb{I}(h(x)\neq f(x))}_{\text{\highlight{\text{关键项}}}} \label{eq:keyterm}
    \end{align}
    \end{proofstep}
    
    \begin{proofstep}[计算关键项]\label{step:key}
    对于二分类问题,每个 $x$ 处的 $f(x)$ 有等概率取 0 或 1:
    \[
    \sum_f \mathbb{I}(h(x)\neq f(x)) = \frac{1}{2} \cdot 2^{|\mathcal{X}|} = 2^{|\mathcal{X}|-1}
    \]
    \end{proofstep}
    
    \begin{proofstep}[最终化简]\label{step:final}
    代入关键项并利用 $\sum_h P(h|X,\mathcal{L}_a) = 1$:
    \[
    \text{原式} = 2^{|\mathcal{X}|-1} \sum_{x \in \mathcal{X}- X} P(x)
    \]
    该结果与算法 $\mathcal{L}_a$ 无关,故对任意 $\mathcal{L}_a, \mathcal{L}_b$:
    \[
    \sum_f E_{ote}(\mathcal{L}_a|X,f) = \sum_f E_{ote}(\mathcal{L}_b|X,f)
    \]
    \end{proofstep}
\end{proof}
由\ref{thm:nfl}可知,脱离具体问题,空谈"什么学习算法最好"是毫无意义的
\subsection{发展历程}
推理期:1950s-1970s--符号知识,演绎推理

知识期:1970s中期--符号知识,领域知识

学习期:1980s--机器学习,归纳逻辑程序设计(Inductive Logic Programming)

统计学习:1990s中期--向量机(Support Vector Machine),核方法(Kernel Methods)

深度学习:2000s--神经网络

\subsection{应用现状}
信息科学,自然科学\dots

\subsection{阅读材料}
\subsubsection{机器学习}
国际会议:ICML,NIPS,COLT,ECML(Europe),ACML(Asia)

国际期刊:JMLR,ML

国内会议:CCML,MLA

\subsubsection{人工智能}
国际会议:IJCAI,AAAI

国际期刊:AI,JAIR

\subsubsection{数据挖掘}
国际会议:KDD,ICDM

国际期刊:ACM-TKDD,DMKD

\subsubsection{计算机视觉}
CVPR(会议),IEEE-TPAMI(期刊)

\subsubsection{神经网络}
期刊:NC,IEEE-TNNLS
\subsubsection{统计学}
期刊:AS

\subsection{习题}
\subsubsection{表1.1中若只包含编号为1和4的两个样例,试给出相应的版本空间.}
\begin{figure}[H]
    \centering
    \includegraphics[width=0.8\textwidth]{static/images/西瓜数据集.png}
    \label{table:watermelon-data}
    \caption{表1.1西瓜数据集}
\end{figure}
解:
\begin{table}[ht]
    \centering
    \caption{\text{版本空间表}}
    \label{tab:logic-expressions}
    \begin{tabular}{cccc}
    \toprule
    \text{色泽} & \text{根蒂} & \text{敲声} & \text{逻辑表达式} \\
    \midrule
    \text{青绿} & \text{瓣缩} & \text{浊响} & $(\text{色泽} = \text{青绿}) \land (\text{根蒂} = \text{瓣缩}) \land (\text{敲声} = \text{浊响})$ \\
    \text{青绿} & \text{瓣缩} & * & $(\text{色泽} = \text{青绿}) \land (\text{根蒂} = \text{瓣缩})$ \\
    \text{青绿} & * & \text{浊响} & $(\text{色泽} = \text{青绿}) \land (\text{敲声} = \text{浊响})$ \\
    * & \text{瓣缩} & \text{浊响} & $(\text{根蒂} = \text{瓣缩}) \land (\text{敲声} = \text{浊响})$ \\
    \text{青绿} & * & * & $(\text{色泽} = \text{青绿})$ \\
    \bottomrule
    \end{tabular}
\end{table}
\subsubsection{估算共有多少种可能的假设}
与使用单个合取式来进行假设表示相比,使用“析合范式”将使得假设空间具有更强的表示能力。例如

好瓜 $\Leftrightarrow\left(\right.$ \textnormal{色泽} $=\star)\wedge($ \textnormal{根蒂} $=$ \textnormal{蜷缩} $\wedge$ \textnormal{敲声} $=\star)$

$\vee\left(\right.$ \textnormal{色泽} $=$ \textnormal{乌黑} $\wedge$\textnormal{ 根蒂} $=\star) \wedge$ \textnormal{敲声} $=$ \textnormal{沉闷} $\left.\right)$,

会把 “(色泽 = 青绿) ∧ (根蒂 = 蜷缩) ∧ (敲声 = 清脆)” 以及 “(色泽 = 乌黑) ∧ (根蒂 = 硬挺) ∧ (敲声 = 沉闷)” 都分类为 “好瓜”。
若使用最多包含 k 个合取式的析合范式来表达表 1.1 西瓜分类问题的假设空间,试估算共有多少种可能的假设。

解:色泽包含两种情况(青绿,乌黑),三种选择(青绿,乌黑,*);\\ 根蒂(蜷缩,稍蜷,硬挺)共4种选择;\\ 敲声(浊响/沉闷/清脆)共4种选择;
\par 除了不能存在(*,*,*)的组合,共包含$ 3 \times 4 \times 4 -1 = 47$种组合.\\ 题目求k个合取式的所有可能的组合之和,则有
\begin{equation}
    \boxed{\sum_{i}^{k} \binom{47}{i}}
\end{equation}
\subsubsection{若数据包含噪声,则假设空间中有可能不存在与所有训练样本都一致的假设。在此情形下,试设计一种归纳偏好用于假设选择。}
解:刚入门,可能无法正确回答此问题,但存在的方法应该有权重噪声注入/梯度稳定性惩罚/稀疏性偏好/集成鲁棒性.

\subsubsection{试证明“没有免费的午餐定理”仍成立。}
本章 1.4 节在论述“没有免费的午餐”定理时,默认使用了“分类错误率”作为性能度量来对分类器进行评估。若换用其他性能度量 \( \ell \),则式(1.1)将改为

\[ E_{ote}(\mathcal{L}_a | X, f) = \sum_h \sum_{x \in \mathcal{X} - X} P(x) \ell(h(x), f(x)) P(h | X, \mathcal{L}_a) \]

试证明“没有免费的午餐定理”仍成立。
\begin{proof}
    \begin{proofstep}
        性能度量虽发生了改变,目标函数仍然均匀分布,总误差表达式为:
        \begin{align*}
            E_{ote}(\mathcal{L}_a | X, f) &= \sum_h \sum_{x \in \mathcal{X} - X} P(x) \ell(h(x), f(x)) P(h | X, \mathcal{L}_a)\\
            &=\sum_{x \in \mathcal{X}-X}P(x) \sum_h P(h | X, \mathcal{L}_a) \sum_f \ell(h(x), f(x))
        \end{align*}
    \end{proofstep}
    \begin{proofstep}
       由\ref{thm:nfl}证明过程可知:$\sum_{f} f(x)=2^{|\mathcal{X}|-1}$
       \begin{equation}
        \therefore 
        \sum_f \ell(h(x), f(x)) = 2^{|\mathcal{X}|-1} \left[ \ell(h(x), 0) + \ell(h(x), 1) \right].
       \end{equation}
    \end{proofstep}
    \begin{proofstep}
        \begin{equation}
            \because P(h(x) = 0 | X, \mathcal{L}_a) = P(h(x) = 1 | X, \mathcal{L}_a) = \frac{1}{2}.
            \therefore  \sum_h P(h | X, \mathcal{L}_a) \left[ \ell(h(x), 0) + \ell(h(x), 1) \right] = \frac{1}{2} \left[ \ell(0, 0) + \ell(0, 1) + \ell(1, 0) + \ell(1, 1) \right].
        \end{equation}

        此结果与算法 \(\mathcal{L}_a\) 无关。
    \end{proofstep}
    \begin{equation}
        \sum_f E_{\text{ote}}(\mathcal{L}_a | X, f) = 2^{|\mathcal{X}|-1} \cdot \frac{1}{2} \sum_{x \in \mathcal{X} - X} P(x) \left[ \ell(0, 0) + \ell(0, 1) + \ell(1, 0) + \ell(1, 1) \right].
    \end{equation}
\end{proof}
无论选择何种性能度量 \(\ell\),只要真实目标函数 \( f \) 在所有可能的函数上均匀分布,无免费午餐定理仍然成立。算法的平均性能仅由数据分布和度量 \(\ell\) 的对称性决定,而与算法本身的设计无关。
\subsubsection{试述机器学习能在互联网搜索的哪些环节起什么作用。}
机器学习贯穿搜索的全流程,从理解用户意图到动态优化结果,其核心价值在于通过数据驱动的方式提升搜索效率、准确性和个性化程度。知识库中提到的超参数优化(如随机搜索)、分布式表示(如协同过滤)等技术均为此提供了方法论支持。机器学习贯穿搜索的全流程,从理解用户意图到动态优化结果,其核心价值在于通过数据驱动的方式提升搜索效率、准确性和个性化程度。知识库中提到的超参数优化(如随机搜索)、分布式表示(如协同过滤)等技术均为此提供了方法论支持。
\section{模型评估与选择}
\subsection{经验误差与拟合}
\myitem 错误率:m个样本中有a个样本分类错误,则错误率$E=a/m$.
\myitem 精度:$1-a/m$
\myitem 误差:学习器的实际预测输出与样本的真实输出之间的差异;训练集上的叫\underline{训练误差}/\underline{经验误差},新样本上的叫\underline{泛化误差};
\myitem 过拟合(更容易遇到的情况):将个例的特殊性错误地视为样本的普遍性;欠拟合:没有学好.
\par 过拟合无法避免(N=NP问题尚未证明),只能缓解或减小其风险
\myitem 模型选择:在多种学习算法中,选择合适的算法中的合适的参数配置.
\subsection{评估方法}
\myitem 测试集测试学习器对新样本的判断力,将测试误差近似于泛化误差.\underline{测试集尽可能的不出现在训练集中}.(老师更希望学生有举一反三的能力)
\myitem \textbf{如果只有一个数据集,既要训练又要测试,如何解决?}
\subsubsection{留出法}
直接将数据集D划分为两个互斥的集合,其中一个集合作为训练集S,另一个作为测试集T.在S上训练出模型后用T来评估其测试误差.
\myitem 训练/测试集的划分要尽可能保持数据分布的一致性(分类任务时,至少要保持样本的类别比例相似)采样角度上看,则保留类别比例的采样方式通常称为\underline{分层采样}\par
例如:通过对D进行分层采样而获得含70\%样本的训练集S和含30\%样本的测试集T,
若D包含500个正例、500个反例,则分层采样得到的S应包含350个正例、350个反例,而T则包含150个正例和150个反例;
若S、T中样本类别比例差别很大,则误差估计将由于训练/测试数据分布的差异而产生偏差.
\myitem 单次使用留出法得到的估计结果往往不够稳定可靠,在使用留出法时,一般要采用若干次随机划分、重复进行实验评估后取平均值作为留出法的评估结果.\par
例如进行100次随机划分,每次产生一个训练/测试集用于实验评估,100次后就得到100个结果,而留出法返回的则是这100个结果的平均.
\myitem 局限性:若令训练集S包含绝大多数样本,则训练出的模型可能更接近于用D训练出的模型,但由于T比较小,评估结果可能不够稳定准确;
若令测试集T多包含一些样本,则训练集S与D差别更大了,被评估的模型与用D训练出的模型相比可能有较大差别,从而降低了评估结果的保真性(fidelity).这个问题没有完美的解决方案,常见做法是将大约2/3~4/5的样本用于训练,剩余样本用于测试.
\subsubsection{交叉验证法}
将数据集分为k个相似的子集,每一折交叉验证时,k-1为训练集,剩余的一个为测试集,共k次不同的测试集,为k折交叉验证法.
交叉验证法同样需要进行随机划分,例如10次10折交叉验证法和100次留出法都需要进行100次训练/测试.
\par \textbf{\heiti 留一法(LOO)},数据集共m个,当k=m时,为留一法.训练出来的模型会更准确,但计算开销较大.
\subsubsection{自助法(Bootstrapping)}
有放回的重复随机抽样,执行m次得到m个样本的数据集$D'$.其中,某些样本始终不会被抽中的概率为
$\lim_{m->\infty}(1-\frac{1}{m})^m\mapsto\frac{1}{e}\approx 0.368$
自助法适用于小样本.
\subsubsection{小结}
\begin{table}[ht]
\centering
\begin{tabular}{lcccc}
\toprule
\textbf{方法} & \textbf{数据划分方式} & \textbf{数据利用率} & \textbf{计算成本} & \textbf{适用场景} \\
\midrule
留出法 & 单次划分训练集/测试集 & 低 & 低 & 大数据快速验证 \\
交叉验证 & 多次划分(\(k\)个子集轮流验证) & 高 & 中 & 中等数据模型调优 \\
留一法 & 每次留1个样本验证 & 最高 & 高 & 小样本高精度评估 \\
自助法 & 有放回重采样生成多样本集 & 灵活 & 中 & 小样本统计推断、不确定性估计 \\
\bottomrule
\end{tabular}
\end{table}

\textbf{\heiti 选择建议}
\begin{itemize}[noitemsep]
    \item 数据量大且需快速验证:留出法。
    \item 中等数据调参:\(k\)-折交叉验证(如5折或10折)。
    \item 小样本高精度需求:留一法(但需权衡计算成本)。
    \item 统计量分布估计或小样本:自助法。
\end{itemize}
\subsubsection{调参与最终模型}
调参对最终模型有关键性影响,通过验证集来评估模型的好坏。
\newpage
\subsection{性能度量}
使用不同的性能度量会导致不同的评判结果。
比如,回归任务最常用的性能度量是“均方误差”:
\begin{equation}
    E(f;D)=\frac{1}{m}\sum_{i=1}{m}(f(x_i)-y_i)^2
\end{equation}
更一般的,对于数据分布D和概率密度函数$p(\cdot)$,均方误差可表示为:
\begin{equation}
    E(f;D)=\int_{x\sim D} (f(x)-y)^2 p(x)dx
\end{equation}
\subsubsection{错误率与精度}
对于样例集D,分类错误率定义为:
\begin{equation}
    E(f;D)=\frac{1}{m}\sum_{i=1}^{m} \mathbb{I}(f(x_i)\neq y_i)
\end{equation}

精度定义为:
\begin{align}
   \text{acc}(f;D) &= 1 - E(f;D) \nonumber \\
    &= \frac{1}{m} \sum_{i=1}^{m} \mathbb{I}(f(x_i) = y_i)
\end{align}

更一般的,对于数据分布D和概率密度函数$p(\cdot)$,错误率可表示为:
\begin{equation}
    E(f;D)=\int_{x\sim D} \mathbb{I}(f(x)\neq y) p(x)dx
\end{equation}

精度可表示为:
\begin{align}
    \text{acc}(f;D) &= 1 - E(f;D) \nonumber \\
    &= \int_{x\sim D} \mathbb{I}(f(x) = y) p(x)dx
\end{align}
\subsubsection{查准率、查全率与F1值}
查准率(Precision)和查全率(Recall)是分类任务中常用的性能度量,尤其在处理不平衡数据集时非常重要。
查准率亦称准确率,表示预测为正例的样本中实际为正例的比例:
\begin{equation}
    \text{Precision} = \frac{\text{TP}}{\text{TP} + \text{FP}}  
\end{equation}
其中,TP表示真正例(True Positives),即被正确预测为正例的样本数;FP表示假正例(False Positives),即被错误预测为正例的样本数。
查全率亦称召回率,表示实际为正例的样本中被正确预测为正例的比例:
\begin{equation}
    \text{Recall} = \frac{\text{TP}}{\text{TP} + \text{FN}}
\end{equation}
其中,TP表示真正例,FN表示假负例(False Negatives),即被错误预测为负例的正例样本数。
查准率和查全率之间通常存在权衡关系:提高查准率可能会降低查全率,反之亦然。P-R曲线(Precision-Recall Curve)可以帮助可视化这种权衡关系。
如图\ref{fig:pr_curve}所示,P-R曲线展示了不同阈值下查准率和查全率的变化情况。
\begin{figure}[H]
    \centering
    \includegraphics[width=0.6\textwidth]{static/images/P-R曲线图.png}
    \caption{P-R曲线示例}
    \label{fig:pr_curve}
\end{figure}
虽然BEP是查准率=查全率时的点,但在实际应用中,查准率和查全率通常不会相等,因此需要综合考虑两者的平衡。
F1值是查准率和查全率的调和平均数,常用来综合评估分类器的性能:
\begin{align}
    F1=\frac{2 \times P \times R}{P+R} \nonumber \\
    &= \frac{2 \times TP}{\text{样例总数}+TP-TN}
\end{align}
但在实际应用中,查准率和查全率的权衡关系通常需要根据具体任务和数据集来调整。所以F1的一般形式--- $F_\beta$,加权调和平均数能够根据任务需求调整查准率和查全率的权重:
\marginpar{与算术平均数和几何平均数相比,调和平均数更重视较小值}
\begin{equation}
    F_\beta = \frac{(1+\beta^2) \times P \times R}{\beta^2 \times P + R}
\end{equation}

对于有多个二分类混淆矩阵的多分类任务,可以使用宏平均(Macro-Averaging)和微平均(Micro-Averaging)来计算整体的查准率、查全率和F1值。
宏平均是对每个类别单独计算查准率和查全率,然后取平均值:
\begin{align}
    P_{macro} &= \frac{1}{n} \sum_{i=1}^{n} P_i \nonumber \\
    R_{macro} &= \frac{1}{n} \sum_{i=1}^{n} R_i \nonumber \\
    F1_{macro} &= \frac{2 \times P_{macro} \times R_{macro}}{P_{macro} + R_{macro}}
\end{align}

微平均则是将所有类别的TP、FP和FN加总后计算查准率和查全率:
\begin{align}
    P_{micro} &= \frac{\sum_{i=1}^{n} TP_i}{\sum_{i=1}^{n} TP_i + \sum_{i=1}^{n} FP_i} \nonumber \\
    R_{micro} &= \frac{\sum_{i=1}^{n} TP_i}{\sum_{i=1}^{n} TP_i + \sum_{i=1}^{n} FN_i} \nonumber \\
    F1_{micro} &= \frac{2 \times P_{micro} \times R_{micro}}{P_{micro} + R_{micro}}
\end{align}
\subsubsection{ROC曲线与AUC}
ROC曲线(Receiver Operating Characteristic Curve)是评估二分类模型性能的常用工具,通过绘制真正率(TPR)与假正率(FPR)的关系来展示分类器在不同阈值下的表现。
真正率(TPR)也称为查全率(Recall),表示实际正例中被正确预测为正例的比例:
\begin{equation}
    \text{TPR} = \frac{\text{TP}}{\text{TP} + \text{FN}}
\end{equation}
假正率(FPR)表示实际负例中被错误预测为正例的比例:
\begin{equation}
    \text{FPR} = \frac{\text{FP}}{\text{FP} + \text{TN}}    
\end{equation}
ROC曲线通过改变分类阈值,计算不同阈值下的TPR和FPR,从而绘制出一条曲线。理想情况下,ROC曲线应尽可能接近左上角(TPR=1, FPR=0),表示模型能够正确识别所有正例且不误判负例。

AUC(Area Under the Curve)是ROC曲线下的面积,表示模型的整体性能。AUC值介于0和1之间,值越大表示模型性能越好。AUC=0.5表示模型没有区分能力,相当于随机猜测;AUC=1表示完美分类器。

\begin{figure}[H]
    \centering
    \includegraphics[width=0.6\textwidth]{static/images/ROC曲线与AUC.png}
    \caption{ROC曲线示例}
    \label{fig:roc_curve}
\end{figure}
AUC可以通过计算ROC曲线下的积分来获得,常用的计算方法包括梯形法则和蒙特卡洛积分等。估算公式为:
\begin{equation}
    AUC = \frac{1}{2} \sum_{i=1}^{n-1} (FPR_{i+1} - FPR_i)(TPR_{i+1} + TPR_i)
\end{equation}

形式上看,AUC考虑的是样本预测的排序质量,因此它与排序误差紧密相连,给定$m^+$个正例和$m^-$个反例,令$D^+$和$D^-$分别为正例和反例的样本集

则loss可以表示为:
\begin{equation}
    \ell_{rank} = \frac{1}{m^+ m^-} \sum_{x^+ \in D^+} \sum_{x^- \in D^-}\big( \mathbb{I}(f(x^+) <f(x^-))+\frac{1}{2}\mathbb{I}(f(x^+) = f(x^-))\big)
\end{equation}

容易看出,$\ell_{rank}$对应的是ROC曲线上的面积,故AUC可以表示为:
\begin{equation}
    AUC = 1 - \ell_{rank}
\end{equation}
\subsubsection{代价敏感错误率与代价曲线}
非均等代价: 在某些应用中,不同类型的错误可能具有不同的代价,例如在医疗诊断中,漏诊(假阴性)可能比误诊(假阳性)更严重.
代价敏感错误率: 在这种情况下,需要引入代价敏感错误率来衡量模型的性能,定义为:
\begin{equation}
    E_{f;D;cost} = \frac{1}{m} \big(\sum_{x_i \in D^+} \mathbb{I}(f(x_i) \neq y_i) \cdot cost_{FP} + \sum_{x_i \in D^-} \mathbb{I}(f(x_i) \neq y_i) \cdot cost_{FN}\big)
\end{equation}
其中, $cost_{FP}$和$cost_{FN}$分别表示假正例和假负例的代价.
代价曲线: 代价曲线是代价敏感错误率随分类阈值变化的图形表示,可以帮助选择最优的分类阈值以最小化总代价.
其中,横轴是取值为[0,1]的正例概率代价:
\begin{equation}
    P(+)cost=\frac{p \cdot cost_{FP}}{p \cdot cost_{FP} + (1-p) \cdot cost_{FN}}
\end{equation}
其中,p为正例的概率,纵轴是取值为[0,1]的归一化代价:
\begin{equation}
    cost_{norm}=\frac{FNR \cdot p \cdot cost_{FP} + FPR \cdot (1-p) \cdot cost_{FN}}{p \cdot cost_{FP} + (1-p) \cdot cost_{FN}}
\end{equation}
\begin{figure}[H]
    \centering
    \includegraphics[width=0.6\textwidth]{static/images/代价曲线与期望总体代价.png}
    \caption{代价曲线示例}
    \label{fig:cost_curve}
\end{figure}
\subsection{比较检验}
\subsubsection{假设检验}


\chapter{线性模型}
\section{基本形式}
线性模型是统计学中一种重要的模型形式,通常用于描述因变量与一个或多个自变量之间的线性关系。其向量形式可以表示为:
\begin{equation}
\mathbf{y} = \beta^\top \mathbf{X} + \epsilon
\end{equation}
其中,$\mathbf{y}$ 是因变量的观测值向量,$\mathbf{X}$ 是自变量的设计矩阵,$\beta$ 是待估计的参数向量,$\epsilon$ 是误差项。
\section{线性回归}
最小二乘法是线性模型中最常用的参数估计方法,其目标是最小化观测值与模型预测值之间的平方差:
\begin{equation}
\hat{\beta} = \arg\min_{\beta} \|\mathbf{y} - \mathbf{X} \beta\|^2
\end{equation}
其中, $\hat{\beta}$ 是参数的估计值。$arg\min$ 表示取使目标函数最小化的参数值。

当$\mathbf{X}^\top \mathbf{X}$ 不是满秩时,最小二乘法的解可能不唯一,此时可以使用岭回归等方法进行正则化,以获得更稳定的参数估计。

广义线性模型(GLM)是线性模型的推广,允许因变量服从非正态分布,并通过链接函数将期望值与线性预测器联系起来。GLM的形式为:
\begin{equation}
    g(\mathbb{E}[Y|\mathbf{X}]) = \beta^\top \mathbf{X}
\end{equation}
其中,$g(\cdot)$ 是链接函数,$\mathbb{E}[Y|\mathbf{X}]$ 是因变量的条件期望。例如,当$g(\cdot)=\log(\cdot)$时,GLM为对数线性模型,适用于计数数据;当$g(\cdot)=\text{logit}(\cdot)$时,GLM为逻辑回归模型,适用于二分类数据。
\begin{lstlisting}[language=Matlab, caption=线性回归的Matlab代码,label=lst:linear_regression]
% 线性回归的Matlab代码示例
clc,clear;
% 生成模拟数据
n = 100; % 样本数量
X = [ones(n, 1), randn(n, 2)]; % 自变量矩阵,第一列为常数项(截距)
beta_true = [1; 2; 3]; % 真实参数
y = X * beta_true + randn(n, 1) * 0.5; % 添加噪声生成因变量

% 拟合线性回归模型
mdl = fitlm(X, y);

% 输出回归结果
disp('参数估计:');
disp(mdl.Coefficients.Estimate);
disp(['R方: ', num2str(mdl.Rsquared.Ordinary)]);
disp(['调整后的R方: ', num2str(mdl.Rsquared.Adjusted)]);
disp(['均方误差: ', num2str(mean(mdl.Residuals.Raw.^2))]);

% 绘制回归结果及诊断图
figure;
plot(mdl);
title('线性回归拟合图');

% 输出结果
%{
参数估计:
         0
    1.0325
    2.0137
    3.0050
R方: 0.98584
调整后的R方: 0.98555
均方误差: 0.22526
%}
\end{lstlisting}
\begin{figure}[H]
    \centering
    \includegraphics[width=0.8\textwidth]{./static/images/线性回归图.png}
    \caption{线性回归拟合结果}
    \label{fig:linear_regression_fit}
\end{figure}
\section{对数几率回归}
对数几率回归(Logistic Regression)是一种广义线性模型,主要用于二分类问题。其模型形式为:
\begin{equation}
    P(Y=1|\mathbf{X}) = g(\beta^\top \mathbf{X}) = \frac{1}{1 + e^{-\beta^\top \mathbf{X}}}
\end{equation}
其中,$P(Y=1|\mathbf{X})$ 是在给定自变量$\mathbf{X}$的条件下,因变量$Y$取值为1的概率,$g(\cdot)$ 是sigmoid函数。

\end{document}