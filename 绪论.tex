\section{绪论}
\subsection{引言}
略
\subsection{基本术语}
\begin{enumerate}[label=(\roman*)]
    \item 样本/示例-sample/instance
    \item 训练集-training data
    \item 测试集-testing data
    \item 标记-label 
    \item 样例-example:拥有label的instance
    \item 泛化-generalization
\end{enumerate}
\subsection{假设空间}
\subsubsection{科学推理}
\begin{itemize}
    \item 归纳--induction
    从具体的事实归结出一般性规律
    \item 演绎--deduction
    从基础原理推演出具体状况
\end{itemize}
\subsubsection{归纳学习--inductive learning}
\begin{itemize}
    \item 广义归纳学习
    \item 狭义归纳学习--概念学习
    eg:布尔概念学习
\end{itemize}
\subsubsection{版本空间--version space}
即存在着一个与训练集一致的"假设集合"
\subsection{归纳偏好}




